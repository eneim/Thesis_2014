%\chapter{Numerical Examples}
\chapter{関連研究}

\section{Crowdsourcing Linked Open Data for Disaster Management}

\cite{cite:opendata}では、近年、災害情報の管理がデータ工学の重要な問題とみなして、災害データ蓄積が巨大な革命であるとわかった.
災害対象者または全世界の人々が災害情報をWEB上で共有することが増えている.
しかし、情報の増加量や時間とともに、それらの情報がよく構造化されていないことがわかった.
そこで、RDF形式で記述されたLinked Open Dataというタームが構成化の問題を解決できるだとJens Ortmannらが論じた.
この研究で、 Ushahidi Haitiプラットフォームを使用することで、 災害情報をLinked Open
Dataによる公開・共有するための必要条件や環境設計について論じた.

\section{LUBM - Lehigh University Benchmark}

RDF/OWLデータの実験ベンチマークについてはGuoらの研究があり\cite{cite:lubm}、LUBMというベンチマークが開発されていた.
LUBMでは、大学情報やその関係者(教授・学生など)に対するOWLデータの生成するパターン計画が提案された.
LUBMが任意なサイズや様々な特徴を持つデータセットを生成することができる.
しかし、現実世界のデータとの違いがあると考えられる.例えば、大学の関係者の種類別による大学の情報へのアクセス範囲が定まれない.
または関係者間の関係(指導教員の関係など)の使用シナリオを挙げないことがわかった.

\section{その他}

その他、災害・避難情報に含まれる個人情報など共有範囲を限定すべき情報を共有する際、
それらのRDF形式で記述されたデータへのアクセス制御を考慮しながら暗号化する手法について、児玉らが研究されている\cite{cite:kodama}.
災害・避難情報を暗号化し、暗号化されたデータへのアクセス許可がユーザが持つアクセスレベルによって定められた.\cite{cite:kodama}では
アクセス許可に対する情報の暗号化・復号化手法について研究されている.

また、災害に発生する多量情報を暗号化することが考えられ、ダットら\cite{cite:dat}が
Map-Reduceによる多量の情報を効率的に暗号化・復号化する手法について研究している.

本稿では、LUBMのデータ生成モデルを参考することで、現実情報に近い、
災害対策の実際問題にあるシナリオを導入したデータセットを生成するベンチマークツールを実装することを目的とした.
