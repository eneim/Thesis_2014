%\chapter{Numerical Examples}
\chapter{関連研究}

\section{Crowdsourcing Linked Open Data for Disaster Management}

近年、災害情報の管理に当たって、災害情報のがデータ工学の重要な問題とみなして、
Linked Open Dataによる災害データ蓄積が巨大な革命であるとJensら\cite{cite:opendata}が論じた。

災害対象者または全世界の人々が災害情報をWEB上で共有することが増えているとわかった。
しかし、情報の増加量や時間とともに、それらの情報がよく構造化されていないことがある。
そこで、RDF形式で記述されたLinked Open Dataというタームが構成化の問題を解決できるだと
Jensら\cite{cite:opendata}が論じた。 

この研究では、 Ushahidi Haitiプラットフォームを使用することで、 災害情報をLinked Open Data
による公開・共有するための必要条件や環境設計について考察した。そこで、Linked Open Dataが災害情報の
構成化問題やセマンティック情報交換可能問題を解決できることを判定した。

\section{LUBM - Lehigh University Benchmark}

LUBM(Lehigh University Benchmark)がRDF/OWLデータのベンチマークである。
LUBMでは、大学情報(大学・学部・授業)やその関係者(教授・学生など)に対するOWLデータの生成するパターン計画が提案された。
LUBMが任意なサイズや様々な特徴を持つデータセットを生成することができる。
\cite{cite:lubm}でGuoらがLUBMを用いることで、RDF/OWLストレージの評価を行った。それより、実際にあるデータを扱う時、適切なシステム
を設計することについて論じた。

\section{その他}

災害・避難情報に含まれる個人情報など共有範囲を限定すべき情報を共有する際、
それらのRDF形式で記述されたデータへのアクセス制御を考慮しながら暗号化する手法について、児玉らが研究されている\cite{cite:kodama}。
災害・避難情報を暗号化し、暗号化されたデータへのアクセス許可がユーザが持つアクセスレベルによって定められた。\cite{cite:kodama}では
アクセス許可に対する情報の暗号化・復号化手法について研究されている。

また、災害に発生する多量情報を暗号化することが考えられ、ダットら\cite{cite:dat}が
Map-Reduceによる多量の情報を効率的に暗号化・復号化する手法について研究している。

Jensらの研究\cite{cite:opendata}を始め、RDFの適用可能性により、RDF形式で記述した災害情報・避難場所情報管理システムの
必要性がわかり、現実に近い避難場所情報を用いることでRDF情報管理システムを評価することは本研究のモチベーションである。そうするために、
LUBMのようなベンチマークによるデータ生成ツールを使用することで実現できると考える。

しかしながら、LUBMで生成したデータの中に、個人情報や、関係者間の関係が不足だと考え、第\ref{info_usage}章に記述した実際の使用シナリオに適用可能性が低い。
さらに、避難場所情報が特殊な情報であり、LUBMが生成したデータのままで適用できないことがわかった。

本研究では、LUBMのデータ生成モデルを参考することで、RDF形式で記述した避難場所情報を含め、
関係者間の関係や実際にある使用シナリオを再現できるベンチマークを構成する。
また、生成したデータのもとで、従来の災害対策のための情報管理システムを評価するために、
現実に近い情報とその情報にたいするクエリを構成できることを目指している。

第\ref{sibm_exp}章から、本研究で提案したSIBM - Shelter Information Benchmarkについて説明する。
