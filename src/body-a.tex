%\chapter{Main Results}
\chapter{避難場所に置ける情報の管理・使用例}
\label{info_usage}

避難所は、区市町村があらかじめ指定している避難施設で、災害時等に区市町村長が開設・
管理・運営し、避難者に安全・安心の場として提供することを目的としている。

大規模な災害発生時に、体育館等の施設を避難所として開放し、大量の避難者を受け入れる上では、
避難者情報を管理するシステムが求められる。避難場所情報に対する様々な使用目的があり、その中によく考えられるのは情報の検索や情報の集約である。

\section{避難場所情報の検索}

災害発生時、避難者情報や避難場所で作業する人の情報などの情報を検索することがある。避難場所情報の検索についていくつかの例がある。

\begin{enumerate}
	\item 災害発生時、その周辺の近くに避難することが考えられ、適切に避難できるために、位置情報による避難場所を検索することがある。
	\item 避難する時、大量の人々が避難先へ水平移動をすると、避難先の収容量を超えてしまったり、避難路における混雑が発生したりすることがある。
	そこで、避難場所の収容量やその他の関連情報を検索することがある。
	\item 避難する際、個人情報も含め、避難者の状況や避難する場所などの情報を検索することが多い。
	\item 災害発生のため、ある家族の人たちが違う避難場所で避難することになり、家族関係による検索の適用で、それらの家族の情報を検索場合がある。
	または、情報のアクセスすることに対して、他人がアクセス許可を持たない情報でも、親戚宅・知人宅等がアクセスできると安心であるため、そのような関係を
	考慮しながらアクセス許可を制御することがある。
	\item 避難所においては、被災者の視点に立って、高齢者、障害者、乳幼児、妊産婦等の災害時要援護者や、
	女性に対する配慮が必要であることから、年齢・性別・健康状況による検索場合がある。
\end{enumerate}

\section{避難場所における情報集約}

情報を検索するだけでなく、検索した情報の集約、分析などの作業が考えられる。例えば:

\begin{enumerate}
	\item
	災害発生時、世界の国々からの国際的なサポートがある。薬品・食品などを適切に避難場所へ配ることが大切であり、
	避難者の状況・病類による物品・薬品の分配することが考えられる。そのような作業は避難場所全体の情報を検索した数値による行うことになる。
	\item また、避難場所に分配された食品などの物品に対して、なるべく避難者ありうは役員に適切に分配する必要がある。
	例えば、高齢者のための食品では、使用量や種類が大人に対して違うことや、子供の場合も同様に考えられるだろう。その時、
	避難者や役員の年齢・性別情報により、物品の使用を決まることがある。
	\item 避難するさい、物品や食材などの用品が時間とともに増減である。その対応に当たって、避難場所情報の管理システムによる、使用歴史や
	統計などを集約することもある。
\end{enumerate}

本研究では、避難場所に置ける情報をRDF形式で記述したデータの管理システムを位置づけすることで、上記のような実際の使用シナリオを考慮しながら、避難場所情報管理システムの設計や運営に置ける
必要な評価をするためのベンチマークを構成する。そして、上記の例に対するいくつかのクエリを実験することにより、構成したベンチマークを考察する。
