
世界中で毎年、災害による損失や市民生活への影響が問題となっている。災害発生時には、避難情報を元に避難することが必要であるが、
災害発生後に、避難に関する情報やデータの量は急速に増加する。このため、適切な災害・避難情報の管理が求められる。

一方、近年RDF形式で記述されたデータが幅広く使用され、災害情報などを含む様々なデータがRDF形式で保存されることが増えている。
RDF形式を利用すると、情報の細かな管理やアクセス制限を適切に行うことができるが、既存の災害情報のRDF表現では、被災者個人間の関係や
被災者と医者、ボランティアなどの支援者との関係や、どの支援者がどの被災者にどのような作業を行って良いかといった関係を表現できていない。

そこで、本研究では、これらの詳細な関係をRDF形式で記述した災害・避難情報のデータセットSIBMを提案する。
SIBMの目的は、被災地における被災者-支援者間のデータ共有方式の妥当性の評価を可能にすることである。
具体的には、被災者の怪我の程度などの個人情報は医療従事者にのみ開示し、他の被災者には個人情報へのアクセスを許可しないなどの
データ共有方式の性能や秘匿情報保護能力をSIBMによって計測できるようになることを目指す。