%\chapter{Introduction}
\chapter{序論}

全世界には 、毎年災害が沢山発生する。大規模災害の2011年の大震災の他 、台風による災害事故や土砂災害などによる破壊が多いとわかった。
災害が発生するとき、避難することは重要である。避難する際、避難場所情報の管理や、避難する作業に関係のある人々の情報が多量に増加することがある。
それらの情報を管理・アクセスすることが重要な作業である。現実では、避難することに関する情報の中に、避難場所情報、
避難する人の情報、避難場所で作業する人(アシスタント・ボランチアなど)の情報があり、それらの情報を管理・アクセスすることが考えられる。
災害が発生するとき、それらの情報に対してすぐに対応できることが重要である。 そのため、災害対策作業に対して、
現実に近い情報を使用しながら実験することは実際の要求である。

一方、近年には、様々な新たなデータ記述法が開発されている。その中に、RDF形式で記述されたデータが増えている。
RDF形式で記述された避難・災害情報を公開・管理することが考えられている\cite{cite:opendata}。
RDF形式で記述したデータが、情報へのアクセス範囲をさらに細かく設計できることがわかる。避難作業情報の中に、
そのようなアクセス制限モデルが必要であり、共有範囲を限定すべき情報を暗号化することが現実の問題である。
例えば、避難作業の関係者の中に、避難者にみせられない情報や個人情報の秘密性に応じるアクセス範囲の制限が必要と考えられる\cite{cite:kodama}。
また、災害が発生するとき、情報がだんだん増加することがわかる。多量データを暗号化することも考えられる。
そこで、効率的に暗号化する手順が必要である\cite{cite:dat}。

上記の問題に対して、RDF形式で記述した避難する作業に関するデータセットを使用して様々な使用シナリオで実験する要求がある。
本研究では、RDF形式で記述された避難場所情報と避難作業情報を生成するベンチマークツール - SIBM - を実装する。
そして、実装したベンチマークツールの構成、実験環境と実用例について考察する。
