%\chapter{Conclusion}
\chapter{結論・今後の課題}

\section{結論}

世界中に、自然災害による損失や生活への影響を防ぐために、災害対策が重要な問題である。
災害発生時、避難する作業が大切であり、避難する際に関する情報の増加が考えられる。そこで、適切な情報管理システムが求められる。

その一方、近年、災害情報を含め、様々な情報をRDF形式による共有、公開することが多い。そして、RDF形式による、情報の扱いやアクセス制御を
より細かく設計でき、データの管理作業に重要な特徴である。

本研究は、避難場所情報をRDF形式に記述した情報管理システムの評価を位置づけし、災害対策問題に応じる従来の情報管理システムを評価することについて
研究課題とした。そこで、RDF形式で記述した避難場所情報のベンチマークツールを構成することを目的とし、RDFデータ管理システムの評価を始め、
災害対策に応じる実際の避難する作業で発生した使用シナリオをクエリ化することで、現実に近い避難場所情報データセットを生成することを可能にした。

関係者間の関係や避難場所に対する様々な作業を考慮しながら、RDFデータ形式の適用で、従来の求められるシステムへのベンチマークが構成できた。

\section{今後の課題}

構成したベンチマークに対して、いくつかの解決必要な問題が考えられる。

災害発生時、その発生位置から離れる近距離から避難することが考えられる。本研究では、そのようなモデルをもとにして、避難場所を生成することができたが、
避難場所すうの代わりに、災害の強度による避難場所を生成することが現実であろう。そのため、災害強度をパラメーターとして情報を生成することを可能にすることが
今後の課題と設定する。

また、避難場所における事情が時間とともに変動することが事実である。本研究は、そのような変動を再現できず、固定な情報へのアクセスしかできていない。そのため、
情報の歴史や、時間に対する変動な数値に応じる様々な変動に対するベンチマークを構成することを解決すべき課題であると考えている。

最後に、最近よく研究されている多次元情報源やRDFデータにおける各要素の関連(例えば、あるトリプルのSubjectが他のトリプルのObjectとなることなど)がある。
本研究では、そのような関連を人の所属により表現できるが、それに対するクエリなどをつけることはできなかった。従来の管理システムへ向けて、そのような関連を再現できること
が今度の課題とせっていする。
